% Options for packages loaded elsewhere
\PassOptionsToPackage{unicode}{hyperref}
\PassOptionsToPackage{hyphens}{url}
\PassOptionsToPackage{dvipsnames,svgnames,x11names}{xcolor}
%
\documentclass[
  letterpaper,
  DIV=11,
  numbers=noendperiod,
  oneside]{scrreprt}

\usepackage{amsmath,amssymb}
\usepackage{iftex}
\ifPDFTeX
  \usepackage[T1]{fontenc}
  \usepackage[utf8]{inputenc}
  \usepackage{textcomp} % provide euro and other symbols
\else % if luatex or xetex
  \usepackage{unicode-math}
  \defaultfontfeatures{Scale=MatchLowercase}
  \defaultfontfeatures[\rmfamily]{Ligatures=TeX,Scale=1}
\fi
\usepackage{lmodern}
\ifPDFTeX\else  
    % xetex/luatex font selection
\fi
% Use upquote if available, for straight quotes in verbatim environments
\IfFileExists{upquote.sty}{\usepackage{upquote}}{}
\IfFileExists{microtype.sty}{% use microtype if available
  \usepackage[]{microtype}
  \UseMicrotypeSet[protrusion]{basicmath} % disable protrusion for tt fonts
}{}
\makeatletter
\@ifundefined{KOMAClassName}{% if non-KOMA class
  \IfFileExists{parskip.sty}{%
    \usepackage{parskip}
  }{% else
    \setlength{\parindent}{0pt}
    \setlength{\parskip}{6pt plus 2pt minus 1pt}}
}{% if KOMA class
  \KOMAoptions{parskip=half}}
\makeatother
\usepackage{xcolor}
\usepackage[papersize=letter]{geometry}
\setlength{\emergencystretch}{3em} % prevent overfull lines
\setcounter{secnumdepth}{2}
% Make \paragraph and \subparagraph free-standing
\ifx\paragraph\undefined\else
  \let\oldparagraph\paragraph
  \renewcommand{\paragraph}[1]{\oldparagraph{#1}\mbox{}}
\fi
\ifx\subparagraph\undefined\else
  \let\oldsubparagraph\subparagraph
  \renewcommand{\subparagraph}[1]{\oldsubparagraph{#1}\mbox{}}
\fi


\providecommand{\tightlist}{%
  \setlength{\itemsep}{0pt}\setlength{\parskip}{0pt}}\usepackage{longtable,booktabs,array}
\usepackage{calc} % for calculating minipage widths
% Correct order of tables after \paragraph or \subparagraph
\usepackage{etoolbox}
\makeatletter
\patchcmd\longtable{\par}{\if@noskipsec\mbox{}\fi\par}{}{}
\makeatother
% Allow footnotes in longtable head/foot
\IfFileExists{footnotehyper.sty}{\usepackage{footnotehyper}}{\usepackage{footnote}}
\makesavenoteenv{longtable}
\usepackage{graphicx}
\makeatletter
\def\maxwidth{\ifdim\Gin@nat@width>\linewidth\linewidth\else\Gin@nat@width\fi}
\def\maxheight{\ifdim\Gin@nat@height>\textheight\textheight\else\Gin@nat@height\fi}
\makeatother
% Scale images if necessary, so that they will not overflow the page
% margins by default, and it is still possible to overwrite the defaults
% using explicit options in \includegraphics[width, height, ...]{}
\setkeys{Gin}{width=\maxwidth,height=\maxheight,keepaspectratio}
% Set default figure placement to htbp
\makeatletter
\def\fps@figure{htbp}
\makeatother

\KOMAoption{captions}{tableheading}
\makeatletter
\@ifpackageloaded{bookmark}{}{\usepackage{bookmark}}
\makeatother
\makeatletter
\@ifpackageloaded{caption}{}{\usepackage{caption}}
\AtBeginDocument{%
\ifdefined\contentsname
  \renewcommand*\contentsname{Table of contents}
\else
  \newcommand\contentsname{Table of contents}
\fi
\ifdefined\listfigurename
  \renewcommand*\listfigurename{List of Figures}
\else
  \newcommand\listfigurename{List of Figures}
\fi
\ifdefined\listtablename
  \renewcommand*\listtablename{List of Tables}
\else
  \newcommand\listtablename{List of Tables}
\fi
\ifdefined\figurename
  \renewcommand*\figurename{Figure}
\else
  \newcommand\figurename{Figure}
\fi
\ifdefined\tablename
  \renewcommand*\tablename{Table}
\else
  \newcommand\tablename{Table}
\fi
}
\@ifpackageloaded{float}{}{\usepackage{float}}
\floatstyle{ruled}
\@ifundefined{c@chapter}{\newfloat{codelisting}{h}{lop}}{\newfloat{codelisting}{h}{lop}[chapter]}
\floatname{codelisting}{Listing}
\newcommand*\listoflistings{\listof{codelisting}{List of Listings}}
\makeatother
\makeatletter
\makeatother
\makeatletter
\@ifpackageloaded{caption}{}{\usepackage{caption}}
\@ifpackageloaded{subcaption}{}{\usepackage{subcaption}}
\makeatother
\makeatletter
\@ifpackageloaded{sidenotes}{}{\usepackage{sidenotes}}
\@ifpackageloaded{marginnote}{}{\usepackage{marginnote}}
\makeatother
\ifLuaTeX
  \usepackage{selnolig}  % disable illegal ligatures
\fi
\usepackage{bookmark}

\IfFileExists{xurl.sty}{\usepackage{xurl}}{} % add URL line breaks if available
\urlstyle{same} % disable monospaced font for URLs
\hypersetup{
  pdftitle={Curriculum vitae},
  pdfauthor={Claire M. Curry},
  colorlinks=true,
  linkcolor={blue},
  filecolor={Maroon},
  citecolor={Blue},
  urlcolor={Blue},
  pdfcreator={LaTeX via pandoc}}

\title{Curriculum vitae}
\author{Claire M. Curry}
\date{2024-07-22}

\begin{document}
\maketitle

\bookmarksetup{startatroot}

\chapter{Career}\label{career}

I am a research librarian, specializing in STEM and quantitative, with
additional experience helping researchers synthesize and analyze data.

\section{Goals}\label{goals}

Where does this show up? As a science librarian at OU Libraries in
concert with my colleagues in the STEM Services team, I have presented
and generated open access content around data analysis, management, and
visualization. We continue to expand these offerings each year. As
current chair of our research workshops committee, I'm leading
improvements to our data and research workshop descriptions to improve
cross-disciplinary appeal, expand our default accessibility
accommodations, and produce content to expand uptake of data and
research better practices to make scholars' lives easier and more
productive. I also value helping new STEM scholars enter librarianship.

\section{Skills and Experiences}\label{skills-and-experiences}

\subsection{Teaching Philosophy}\label{teaching-philosophy}

I use a wide range of software tools to explore interesting research
questions from my own areas of expertise (originally in quantitative
biology), in collaboration with other researchers, and now helping
patrons in my current position as a science librarian. I am experienced
in and enjoy helping other researchers use R and SAS, choose appropriate
statistical techniques, and manage their data.

This is evidenced by my certification as an instructor for Software and
Data Carpentries, co-instructing graduate-level experimental design and
statistics (Section Section~\ref{sec-nrm-quant}) during my time at the
Natural Resources Institute, and helping many patrons and collaborators
with R, SAS, ArcGIS, QGIS, and statistics. This is now part of my
position as a science librarian to help guide patrons to appropriate
software and statistical resources.

\subsubsection{Class instruction}\label{class-instruction}

\subsubsection{Workshops/outreach}\label{workshopsoutreach}

\subsubsection{Consultation}\label{consultation}

In my own research, I have used a wide variety of techniques, from
univariate (including basic statistics to mixed models to power
analyses) to multivariate (including canonical correlation analysis to
PCA to Mantel tests) statistics. I have also spoken on more than one
occasion about data and project management from my experiences managing
my own and others' data during my Ph.D.~and post-doctoral work and
instruct with The Carpentries to promote best practices in data and
coding skills for scientists of all disciplines.

\subsubsection{Mentoring}\label{mentoring}

\subsection{Grant Support}\label{grant-support}

\subsection{Research \& Publishing}\label{research-publishing}

\bookmarksetup{startatroot}

\chapter{Education}\label{education}

\section{Ph.D.~in Ecology and Evolutionary Biology
(Zoology)}\label{ph.d.-in-ecology-and-evolutionary-biology-zoology}

University of Oklahoma (Norman, OK, USA)

\marginnote{\begin{footnotesize}

2008-2014

\end{footnotesize}}

Advisor: Dr.~Michael Patten

\section{B.S. in Biology, Chemistry minor (summa cum
laude)}\label{b.s.-in-biology-chemistry-minor-summa-cum-laude}

University of North Texas (Denton, TX, USA)

\marginnote{\begin{footnotesize}

2005-2007

\end{footnotesize}}

Honors thesis advisor: Dr.~James H. Kennedy.

\bookmarksetup{startatroot}

\chapter{Positions Held}\label{positions-held}

\section{Science Librarian}\label{sec-science-librarian}

University Libraries, University of Oklahoma (Norman, OK, USA)

\marginnote{\begin{footnotesize}

2018-present

\end{footnotesize}}

\subsection{Supervisors}\label{supervisors}

\begin{itemize}
\tightlist
\item
  Karie Antell (2024-present)
\item
  Sarah Robbins (2018-2024)
\end{itemize}

\subsection{Supervisory experience}\label{supervisory-experience}

\begin{itemize}
\tightlist
\item
  Ahmed Sharaf (August 2022-August 2023): Undergraduate student web
  developer. professional skills like note taking, working with
  consultants, writing short project updates, creating and implementing
  project milestones, collaborative use of GitHub. Graduated December
  2023.
\end{itemize}

\section{Post-doctoral research
associate}\label{post-doctoral-research-associate}

Oklahoma Biological Survey, University of Oklahoma (Norman, OK, USA)

\marginnote{\begin{footnotesize}

2016-2018

\end{footnotesize}}

\subsection{Supervisors}\label{supervisors-1}

\begin{itemize}
\tightlist
\item
  PI: Dr.~Eli S. Bridge
\item
  PI: Dr.~Michael A. Patten (2018)
\end{itemize}

\subsection{Supervisory experience}\label{supervisory-experience-1}

\begin{itemize}
\tightlist
\item
  Henry Dang (undergraduate research assistant)
\end{itemize}

\section{Research Associate}\label{research-associate}

Natural Resources Institute, University of Manitoba (Winnipeg, MB,
Canada)

\marginnote{\begin{footnotesize}

2016

\end{footnotesize}}

\subsection{Supervisor}\label{supervisor}

\begin{itemize}
\tightlist
\item
  PI: Dr.~Nicola Koper
\end{itemize}

Promotion from previous position as Post-doctoral Fellow.

\section{Post-doctoral Fellow}\label{post-doctoral-fellow}

Natural Resources Institute, University of Manitoba, (Winnipeg, MB,
Canada)

\marginnote{\begin{footnotesize}

2014-2016

\end{footnotesize}}

\subsection{Supervisor}\label{supervisor-1}

\begin{itemize}
\tightlist
\item
  PI: Dr.~Nicola Koper
\end{itemize}

\subsection{Supervisory experience}\label{supervisory-experience-2}

\begin{itemize}
\tightlist
\item
  Managed three field technicians (summer 2015)
\end{itemize}

\section{U.S. Forest Service
Contractor}\label{u.s.-forest-service-contractor}

Caddo/Lyndon B. Johnson National Grasslands, US Forest Service (Decatur,
TX, USA)

\marginnote{\begin{footnotesize}

March-June 2008

\end{footnotesize}}

\subsection{Supervisor}\label{supervisor-2}

\begin{itemize}
\tightlist
\item
  Alfredo Sanchez, USFS Wildlife Biologist (retired)
\end{itemize}

Conducted a Black-capped Vireo survey and designed a Loggerhead Shrike
survey on LBJ National Grasslands.

\section{Undergraduate research
assistant}\label{undergraduate-research-assistant}

Aquatic Ecology Lab, University of North Texas (Denton, TX, USA)

\marginnote{\begin{footnotesize}

March-August 2007

\end{footnotesize}}

\subsection{Supervisor}\label{supervisor-3}

\begin{itemize}
\tightlist
\item
  PI: Dr.~James Kennedy
\end{itemize}

Sorted aquatic invertebrates in sediment samples, identified
Chironomidae subfamilies and formatted data

\section{Seasonal Field Biologist}\label{seasonal-field-biologist}

Leon River Restoration Project (with Texas A\&M University)

\marginnote{\begin{footnotesize}

May-July 2006

\end{footnotesize}}

\subsection{Supervisors}\label{supervisors-2}

\begin{itemize}
\tightlist
\item
  Andy Campomizzi
\item
  Shannon Farrell
\end{itemize}

Field site: private properties near Gatesville, TX

Point counts of birds, nest searching, fledgling searching, and entering
data for one field season.

\section{Undergraduate research
assistant}\label{undergraduate-research-assistant-1}

Aquatic Ecology Lab, University of North Texas (Denton, TX, USA)

\marginnote{\begin{footnotesize}

October 2005-May 2006

\end{footnotesize}}

\subsection{Supervisor}\label{supervisor-4}

\begin{itemize}
\tightlist
\item
  PI: Dr.~James Kennedy
\end{itemize}

Sorted aquatic invertebrates in sediment samples part-time.

\section{Field technician}\label{field-technician}

Texas Agricultural Experiment Station in Vernon, TX

\marginnote{\begin{footnotesize}

June-August 2005

\end{footnotesize}}

\subsection{Supervisor}\label{supervisor-5}

\begin{itemize}
\tightlist
\item
  PI: Dr.~Dean Ransom
\end{itemize}

Counting birds, nest searching, measuring vegetation, and sampling
insects for one field season.

\section{U.S. Forest Service
Contractor}\label{u.s.-forest-service-contractor-1}

Caddo/Lyndon B. Johnson National Grasslands, US Forest Service (Decatur,
TX, USA)

\marginnote{\begin{footnotesize}

December 2001-February 2002

\end{footnotesize}}

Counting grassland bird species along designated transects on the Lyndon
B. Johnson National Grasslands.

\bookmarksetup{startatroot}

\chapter{Teaching Experience}\label{teaching-experience}

\section{Research Data Workshops}\label{research-data-workshops}

Teach, develop, and modify curricula to enhance researchers' abilities
to process and visualize information.

\marginnote{\begin{footnotesize}

2018-present

Section~\ref{sec-science-librarian}

\end{footnotesize}}

\begin{itemize}
\tightlist
\item
  Developed curricula for multiple workshops

  \begin{itemize}
  \tightlist
  \item
    Smoother Meetings
  \item
    Repetitive Tasks in R
  \item
    ggplot2 in R
  \item
    Troubleshooting Computer Code (collaboration with co-instructor
    Logan Cox)
  \item
    Concepts of Data-Driven Visualization
  \end{itemize}
\item
  Co-developed curricula for multiple workshops

  \begin{itemize}
  \tightlist
  \item
    Managing Research Files
  \item
    Citation management with Zotero
  \item
    Open Science Framework intro
  \end{itemize}
\item
  Instructing from existing curricula

  \begin{itemize}
  \tightlist
  \item
    R for Scientific Programming
  \item
    R for Ecology
  \item
    R for Social Sciences
  \item
    R for GIS

    \begin{itemize}
    \tightlist
    \item
      Including modifications for 2.5 hr two session short version
    \end{itemize}
  \item
    Tidy data/spreadsheets
  \item
    SQL
  \item
    ??
  \item
    Git/GitHub

    \begin{itemize}
    \tightlist
    \item
      Including modifications for 2.5 hr two session short version
    \end{itemize}
  \item
    Bash
  \end{itemize}
\item
  Helping from existing curricula

  \begin{itemize}
  \tightlist
  \item
    OpenRefine
  \item
    Python
  \end{itemize}
\item
  Chairing organizational committee since its inception (2022)

  \begin{itemize}
  \tightlist
  \item
    Responsible for reminding instructors of scheduling
  \item
    Automated emails via Zapier and Trello to instructors to remind them
    and to listserv to announce registration
  \end{itemize}
\item
  Certified Carpentries instructor (as of 2019) as part of Science
  Librarian position
\end{itemize}

\section{Library instruction}\label{library-instruction}

Teach, develop, and scaffold information literacy skills to
undergraduate and graduate-level STEM students

\marginnote{\begin{footnotesize}

2018-present

\end{footnotesize}}

\begin{itemize}
\tightlist
\item
  Develop and deliver one-shot and multiple visit teaching sessions
\item
  Develop information literacy materials for asynchronous learning
\item
  Using ACRL STEM literacy framework
\item
  Deliver materials in-person, hybrid, and virtual
\item
  Teach solo, with a shadow trainee, and in collaboration (for
  interdisciplinary courses)
\item
  20-50 class visits per year teaching information literacy to
  undergraduate and graduate classes in STEM fields
\end{itemize}

\section{Instructor of record}\label{instructor-of-record}

\subsection{Study Design and Quantitative Methods in Natural Resource
Management}\label{sec-nrm-quant}

Cross-listed graduate/undergraduate lab.

\marginnote{\begin{footnotesize}

Winter 2015, winter 2016

\end{footnotesize}}

Co-instructor with Dr.~Nicola Koper

Presenting R code to implement statistical methods; lecture on
multivariate statistics; and grading half of assignments

\subsection{Ecological Dimensions of Natural Resource
Management}\label{ecological-dimensions-of-natural-resource-management}

Graduate level lecture class

\marginnote{\begin{footnotesize}

Fall 2015

\end{footnotesize}}

Sessional instructor: Lecture, guide discussions, grade assignments.

\section{Mentoring}\label{mentoring-1}

\subsection{University of Manitoba
Students}\label{university-of-manitoba-students}

Guidance on project design, sampling, statistical analysis, and use of R
and SAS for Master's (14), undergraduate (1), and Ph.D.~(1) students in
Nicola Koper lab

\marginnote{\begin{footnotesize}

2014-2016

\end{footnotesize}}

Master's (2) and Ph.D.~(1) students at the Natural Resources Institute
and Biology departments.

\subsection{University of Oklahoma
Undergraduates}\label{university-of-oklahoma-undergraduates}

\subsubsection{Alexandra Ainsworth}\label{alexandra-ainsworth}

Independent study (3 credit hours), PI: Michael A. Patten

\marginnote{\begin{footnotesize}

Spring 2012

\end{footnotesize}}

She fed captive birds, assisted with mate choice trial set-up, bird
capture, DNA extraction using Qiagen DNeasy kits, PCR, and agarose gel
electrophoresis.

\subsubsection{Sarai H. Stuart}\label{sarai-h.-stuart}

Volunteer, recently graduated from OU.

\marginnote{\begin{footnotesize}

January-April 2012, January-March 2013

\end{footnotesize}}

Volunteer was a recent OU graduate wanting field experience before
applying to jobs and/or graduate school. She learned about mist-netting,
trapping, banding, setting up mate choice trials, maintaining captive
birds, DNA extraction using Qiagen DNeasy kits, PCR, and agarose gel
electrophoresis. Graduated with Ph.D.~from University of Illinois at
Champaign-Urbana.

\section{Graduate Teaching
Assistantships}\label{graduate-teaching-assistantships}

\subsection{Intro. to Evolution, Ecology, and
Diversity}\label{intro.-to-evolution-ecology-and-diversity}

Undergraduate, majors lab

\marginnote{\begin{footnotesize}

2012 (spring) One semester

\end{footnotesize}}

Professor: Dr.~Phil Gibson

Presentation of weekly lab materials, creating and grading quizzes,
grading lecture exams, proctoring lecture and lab exams, and guiding
students through lab experiments.

\subsection{Entomology}\label{sec-teach-ento}

Cross-listed undergraduate/graduate lab

\marginnote{\begin{footnotesize}

2009-2012 (fall only - total four semesters)

\end{footnotesize}}

Professor: Dr.~Ken Hobson (retired)

Presentation of insect orders and families; creating and grading quizzes
and exam; and leading students on field trips

\subsection{Introductory Zoology Lab}\label{introductory-zoology-lab}

Undergraduate, non-majors lab

\marginnote{\begin{footnotesize}

Fall 2008, fall 2012

\end{footnotesize}}

Professor: Dr.~Penny Hopkins (retired)

Lecturing on the day's lab topic, assisting students with dissections,
and writing, administering, and grading quizzes

\subsection{Principles of Ecology}\label{sec-teach-ecology}

Undergraduate, majors lab

\marginnote{\begin{footnotesize}

2009 (spring \& fall), 2010-2011 (fall only)

\end{footnotesize}}

Professor: Dr.~Ken Hobson (retired)

Discussion of concepts, guiding students through field experiments; and
grading lab reports.

\section{Undergraduate Teaching Assistant
(Volunteer)}\label{undergraduate-teaching-assistant-volunteer}

\subsection{Insect Biology Lab}\label{insect-biology-lab}

Cross-listed undergraduate/graduate lab

\marginnote{\begin{footnotesize}

Fall 2006, fall 2007

\end{footnotesize}}

Professor: Dr.~James Kennedy

Help students with keying insects, assist students on field trips, and
proctor exams and quizzes with the graduate teaching assistants.

\bookmarksetup{startatroot}

\chapter{Publications and
Presentations}\label{publications-and-presentations}

\section{Peer-reviewed}\label{sec-peer}

1. Semenov GA, Curry CM, Patten MA, Weir JT, Taylor SA. Geographically
consistent hybridization dynamics between the {Black-crested} and
{Tufted Titmouse} with evidence of hybrid zone expansion. Ornithology.
2023 {[}accessed 2023 Apr 19{]};140(3):ukad014.
doi:\href{https://doi.org/10.1093/ornithology/ukad014}{10.1093/ornithology/ukad014}

2. Robbins S, Curry C, Schilling A, Tweedy BN. Recruiting, {Hiring}, \&
{On-Boarding Non-MLS Liaison Librarians}: {A Case Study}. Library
Leadership \& Management. 2022 {[}accessed 2022 Mar 14{]};36(1).
doi:\href{https://doi.org/10.5860/llm.v36i1.7490}{10.5860/llm.v36i1.7490}

3. Rockwood JV, Curry CM, Day EA. Distributed expertise, leadership
structure \& efficiency-viability tradeoffs in teams {[}{Poster}{]}. In:
Society for {Industrial} and {Organizational Psychology Annual
Conference}. Seattle, WA, USA; 2022.

4. Patten MA, Barnard AA, Curry CM, Dang H, Loraamm RW. Forging a
{Bayesian} link between habitat selection and avoidance behavior in a
grassland grouse. Scientific Reports. 2021 {[}accessed 2021 Jun
4{]};11(1):2791.
doi:\href{https://doi.org/10.1038/s41598-021-82500-0}{10.1038/s41598-021-82500-0}

5. Gaffin DD, Curry CM. Arachnid navigation -- a review of classic and
emerging models. The Journal of Arachnology. 2020 {[}accessed 2020 Jul
20{]};48(1):1.
doi:\href{https://doi.org/10.1636/0161-8202-48.1.1}{10.1636/0161-8202-48.1.1}

6. Curry CM, Patten MA. Complex spatiotemporal variation in processes
shaping song variation. Behaviour. 2019 {[}accessed 2019 Apr
17{]}:1--26.
doi:\href{https://doi.org/10.1163/1568539X-00003556}{10.1163/1568539X-00003556}

7. Warrington MH, Curry CM, Antze B, Koper N. Noise from four types of
extractive energy infrastructure affects song features of {Savannah
Sparrows}. Condor. 2018 {[}accessed 2017 Dec 14{]};120(1):1--15.
doi:\href{https://doi.org/10.1650/CONDOR-17-69.1}{10.1650/CONDOR-17-69.1}

8. Curry CM, Brisay PGD, Rosa P, Koper N. Noise source and individual
physiology mediate effectiveness of bird songs adjusted to anthropogenic
noise. Scientific Reports. 2018 {[}accessed 2018 Mar 2{]};8(1):3942.
doi:\href{https://doi.org/10.1038/s41598-018-22253-5}{10.1038/s41598-018-22253-5}

9. Curry CM, Ross JD, Contina AJ, Bridge ES. Varying dataset resolution
alters predictive accuracy of spatially explicit ensemble models for
avian species distribution. Ecology and Evolution. 2018 {[}accessed 2018
Dec 7{]};0(0).
doi:\href{https://doi.org/10.1002/ece3.4725}{10.1002/ece3.4725}

10. Curry CM, Antze B, Warrington MH, Brisay PD, Rosa P, Koper N.
Ability to alter song in two grassland songbirds exposed to simulated
anthropogenic noise is not related to pre-existing variability.
Bioacoustics. 2017 Mar {[}accessed 2017 Mar 3{]}.

11. Pipher EN, Curry CM, Koper N. Cattle {Grazing Intensity} and
{Duration Have Varied Effects} on {Songbird Nest Survival} in
{Mixed-Grass Prairies}. Rangeland Ecology \& Management. 2016
{[}accessed 2016 Nov 22{]};69(6):437--443.
doi:\href{https://doi.org/10.1016/j.rama.2016.07.001}{10.1016/j.rama.2016.07.001}

12. Curry CM, Patten MA. Shadow of a doubt: Premating and postmating
isolating barriers in a temporally complex songbird ({Passeriformes}:
{Paridae}) hybrid zone. Behavioral Ecology and Sociobiology.
2016;70(8):1171--1186.
doi:\href{https://doi.org/10.1007/s00265-016-2126-y}{10.1007/s00265-016-2126-y}

13. Koper N, Leston L, Baker TM, Curry C, Rosa P. Effects of ambient
noise on detectability and localization of~avian songs and tones by
observers in grasslands. Ecology and Evolution. 2016 {[}accessed 2016
Jan 20{]};6(1):245--255.
doi:\href{https://doi.org/10.1002/ece3.1847}{10.1002/ece3.1847}

14. Curry CM. An {Integrated Framework} for {Hybrid Zone Models}.
Evolutionary Biology. 2015 {[}accessed 2015 Oct 14{]};42(3):359--365.
doi:\href{https://doi.org/10.1007/s11692-015-9332-9}{10.1007/s11692-015-9332-9}

15. Curry CM, Patten MA. Current and historical extent of phenotypic
variation in the {Tufted} and {Black-crested Titmouse} ({Paridae})
hybrid zone in the southern {Great Plains}. American Midland Naturalist.
2014;171(2):271--300.
doi:\href{https://doi.org/10.1674/0003-0031-171.2.271}{10.1674/0003-0031-171.2.271}

16. Bridge ES, Kelly JF, Bjornen PE, Curry CM, Crawford PHC, Paritte JM.
Effects of nutritional condition on spring migration: Do migrants use
resource availability to keep pace with a changing world? Journal of
Experimental Biology. 2010 {[}accessed 2014 May 6{]};213(14):2424--2429.
doi:\href{https://doi.org/10.1242/jeb.041277}{10.1242/jeb.041277}

17. Curry CM, Kennedy JH. Factors affecting interaction rates in
{\emph{Plathemis}}{ \emph{Lydia}} ({Drury}) ({Anisoptera}:
{Libellulidae}). Odonatologica. 2010;39(1):29--38.

\section{Reports}\label{reports}

Ithaka

\section{Non-peer-reviewed Talks}\label{non-peer-reviewed-talks}

\subsection{Conferences}\label{conferences}

*indicates presenter

\begin{itemize}
\tightlist
\item
  Schilling, A.*, M. Laufersweiler, B. Tweedy, and C.M. Curry. Data
  workshops in support of researchers at the University of Oklahoma. OSU
  -- Coalition for Advancing Digital Research \& Education (CADRE) 2020:
  Stillwater, OK, USA. (Online poster available at:
  https://shareok.org/handle/11244/324832 )
\item
  Schilling, A.*, B. Tweedy*, and C.M. Curry*. 2019. An Open Access Data
  Workshop Curriculum for Researchers. 2019 STEM Librarians South:
  Austin, TX, USA. (Oral presentation.)
\item
  Curry C.M., J.E. Ruyle, and E.S. Bridge*. 2018. ETAG: An Animal
  Behavior Observatory for Radio Frequency Identification Technology.
  2018 International Ornithological Congress, Vancouver, BC, Canada.
  (Poster presentation.)
\item
  Curry, C.M.* and M.A.~Patten. 2018. Thresholds of Avoidance Behavior
  in a Declining Endemic Grouse, the Lesser Prairie-Chicken. 2018
  International Ornithological Congress: Vancouver, BC, Canada. (Oral
  presentation.)
\item
  Curry, C.M.*, M.A.~Patten, and J. Weir. 2018. Spatiotemporal
  structuring and genomic architecture of multiple transects across an
  avian hybrid zone. 2018 American Ornithology: Tucson, AZ, USA. (Oral
  presentation.)
\item
  Curry C.M.*, J.E. Ruyle, and E.S. Bridge. 2018. ETAG: An Animal
  Behavior Observatory for Radio Frequency Identification Technology.
  2018 American Ornithology: Tucson, AZ, USA. (Poster presentation.)
\item
  Curry, C.M.*, J.D. Ross, A. J. Contina, and E.S. Bridge. 2017. Testing
  prediction accuracy and climate change estimation with spatially
  explicit models for grassland birds. 2017 American Ornithology: East
  Lansing, MI, USA. (Oral presentation.)
\item
  Curry, C.M.*, B. Antze, M. Warrington, P. Des Brisay, P. Rosa, and N.
  Koper. 2016. Behavioral responses to anthropogenic noise in two
  species of grassland songbirds in the Canadian mixed-grass prairie.
  North American Ornithological Conference 2016: Washington, DC., USA.
  (Oral presentation.)
\item
  Curry, C.M.*, M.A.~Patten, and J. Weir. 2016. Geographic and genomic
  clines, population structure, and range expansion in a temporally
  complex titmouse (Paridae) hybrid zone. North American Ornithological
  Conference 2016: Washington, DC., USA. (Poster presentation.)
\item
  Curry, C.M.*, B. Antze, M. Warrington, P. Des Brisay, P. Rosa, and N.
  Koper. 2016. Behavioural responses of two grassland songbirds to
  anthropogenic noise pollution. Manitoba Chapter of The Wildlife
  Society Annual General Meeting: Winnipeg, MB. (Oral presentation.)
\item
  Koper, N.*, C.M. Curry, B. Antze, and M. Warrington. 2016. Effects of
  energy infrastructure operating noise on behaviour of Savannah and
  Baird's sparrows. Alberta Chapter of The Wildlife Society, Drumheller,
  Alberta, Canada. (Oral presentation.)
\item
  Curry, C.M.*, P. Des Brisay, B. Antze, H. Nenninger, M. Warrington,
  and N. Koper. 2015. Behavioural response of two grassland songbirds to
  noise pollution from energy development in the Canadian prairie. The
  Wildlife Society Annual Meeting: Winnipeg, MB. (Oral presentation).
\item
  Koper, N.*, J. Bernath-Plaisted, C.M. Curry, B. Antze, M. Warrington,
  H. Nenninger, C. Swider, and P. Rosa. 2015. Effects of oil and gas
  infrastructure and operating noise on grassland songbirds in Alberta.
  The Wildlife Society Annual Meeting: Winnipeg, MB. (Oral
  presentation.)
\item
  Curry, C.M.*, P. Des Brisay, B. Antze, H. Nenninger, M. Warrington,
  and N. Koper. 2015. Behavioural response of two grassland songbirds to
  noise pollution from energy development in the Canadian prairie.
  International Congress for Conservation Biology: Montpellier, France.
  (Five minute speed talk.)
\item
  Koper, N.*, J. Bernath-Plaisted, C.M. Curry, B. Antze, M. Warrington,
  H. Nenninger, and P. Rosa. 2015. Effects of oil and gas infrastructure
  and operating noise on grassland songbirds in Canada. International
  Congress for Conservation Biology: Montpellier, France. (Oral
  presentation).
\item
  Koper, N.*, J. Bernath-Plaisted, C.M. Curry, H. Nenninger, P. Rosa,
  and C. Swider. 2015. Effects of energy infrastructure and operating
  noise on grassland songbirds. Alberta Chapter of the Wildlife Society:
  Edmonton, AB. (Oral presentation).
\item
  Curry, C.M.*, P. Des Brisay, B. Antze, H. Nenninger, and N. Koper.
  2014. Effects of oil and gas infrastructure noise on Baird's Sparrow
  songs. Parks and Protected Areas Research Forum of Manitoba: Winnipeg,
  MB. (Oral presentation).
\item
  Curry, C.M.* and M.A.~Patten. 2014. Evolution of reproductive
  isolation in a temporally complex hybrid zone between Tufted and
  Black-crested Titmice (Paridae). American Ornithologists' Union/Cooper
  Ornithological Society/Society of Canadian Ornithologists joint
  meeting: Estes Park, CO. (Oral presentation).
\item
  Curry, C.M.* and M.A.~Patten. 2013. Sexual and natural selection on
  song across the temporally complex hybrid zone of Tufted and
  Black-crested Titmice (Paridae). American Ornithologists' Union/Cooper
  Ornithological Society joint meeting: Chicago, IL. (Oral
  presentation).
\item
  Curry, C.M.* and M.A.~Patten. 2013. Sexual and natural selection on
  song across a temporally complex passerine hybrid zone. Evolution 2013
  meeting: Snowbird, UT. (Poster presentation).
\item
  Curry, C.M.* and M.A.~Patten. 2012. Evolution of song variation across
  a complex hybrid zone in Tufted and Black-crested titmice. Oklahoma
  Ornithological Society meeting: Oklahoma City, OK. (Oral
  presentation). Won ``Best Oral Presentation by a Graduate Student''.
\item
  Curry, C.M.* and M.A.~Patten. 2012. Evolution of song variation across
  a complex hybrid zone in Tufted and Black-crested titmice. North
  American Ornithological Conference: Vancouver, BC. (Poster
  presentation).
\item
  Curry, C.M.* and M.A.~Patten. 2012. Causes and consequences of song
  variation across a temporally complex passerine hybrid zone. First
  Joint Congress on Evolutionary Biology: Ottawa, ON. (Oral
  presentation).
\item
  Curry, C.M.* and M.A.~Patten. 2011. Causes of song variation across
  the titmouse hybrid zones in Oklahoma and Texas. Oklahoma
  Ornithological Society meeting: Edmond, OK. (Oral presentation).
\item
  Curry, C.M.* and M.A.~Patten. 2011. Song varies across younger and
  older hybrid zones in Black-crested (Baeolophus atricristatus) and
  Tufted (B. bicolor) titmice. Evolution 2011 meeting: Norman, OK. (Oral
  presentation).
\item
  Curry, C.M.* and M.A.~Patten. 2011. Song varies across younger and
  older hybrid zones in Black-crested (Baeolophus atricristatus) and
  Tufted (B. bicolor) titmice. Association of Field
  Ornithologists/Cooper Ornithological Society/Wilson Ornithological
  Society joint meeting: Kearney, NE. (Oral presentation).
\item
  Curry, C.M.* and M.A.~Patten. 2010. Vocal dynamics of a complex avian
  hybrid zone. Oklahoma Ornithological Society meeting: Stillwater, OK.
  (Oral presentation).
\item
  Curry, C.M.* and M.A.~Patten. 2010. Vocal dynamics of a complex avian
  hybrid zone. Association of Field Ornithologists' annual meeting:
  Ogden, UT. (Poster presentation.)
\item
  Curry, C.M.* and M.A.~Patten. 2010. Vocal dynamics of a complex avian
  hybrid zone. Southwestern Association of Naturalists annual meeting:
  Junction, TX. (Poster presentation).
\item
  Curry, C.M.* and J.H. Kennedy. 2008. Factors affecting interaction
  rates and space use of Plathemis lydia (Drury 1773) (Odonata:
  Libellulidae). Southwestern Branch of the Entomological Society of
  America/Society of Southwestern Entomologists joint meeting: Fort
  Worth, TX. (Poster presentation).
\item
  Curry, C.M.* and J.H. Kennedy. 2008. Factors affecting interaction
  rates and space use of Plathemis lydia (Drury 1773) (Odonata:
  Libellulidae). Southwestern Association of Naturalists annual meeting:
  Memphis, TN. (Oral presentation).
\end{itemize}

\subsection{Invited Lectures and
Seminars}\label{invited-lectures-and-seminars}

*indicates presenter

\begin{itemize}
\tightlist
\item
  Curry, C.M.*. 2020. Zero to online in 60 seconds: moving Undergraduate
  Research Day to Open Science Framework. 2020 University Library Day:
  Norman, OK, USA. (Lightning talk via Zoom. Invited presentation.)
\item
  Curry, C.M.*, P. Des Brisay, P. Rosa, and N. Koper. 2017. Efficacy of
  adjusted songs for communication in noise varies with infrastructure
  type and hormone levels. In: Mechanisms underlying avian response to
  energy development (symposium). 2017 American Ornithology: East
  Lansing, MI, USA. (Oral presentation.)
\item
  Koper, N.*, P. Rosa, C.M. Curry, J. Bernath-Plaisted, H. Nenninger,
  and J. Daniel. 2017. Industrial noise and oil wells: effects on birds.
  University of Winnipeg, Winnipeg, Manitoba, Canada.
\item
  Koper, N.*, P. Rosa, C.M. Curry, J. Bernath-Plaisted, M. Warrington,
  and H. Nenninger. 2017. Effects of noisy oil and gas infrastructure
  and anthropogenic noise on abundance, productivity, and behaviour of
  grassland songbirds. Université du Quebéc à Trois-Rivières, Quebec,
  Canada.
\item
  Koper, N.*, P. Rosa, C.M. Curry, J. Bernath-Plaisted, H. Nenninger, M.
  Warrington, and J. Daniel. 2017. Effects of oil and gas infrastructure
  and anthropogenic noise on grassland songbirds. National Wildlife
  Research Centre Seminar Series, Carleton University, Ottawa, Ontario,
  Canada.
\item
  Koper, N.*, P. Rosa, C.M. Curry, J. Bernath-Plaisted, H. Nenninger,
  and J. Daniel. 2017. Effects of noisy petroleum infrastructure on
  grassland songbirds in Alberta, Canada. Entomology Department Seminar
  Series, University of Manitoba, Winnipeg, Manitoba, Canada.
\item
  Koper, N.*, P. Rosa, C.M. Curry, J. Bernath-Plaisted, H. Nenninger,
  and J. Daniel. 2017. Effects of noisy petroleum infrastructure on
  grassland songbirds in Alberta, Canada. St.~George's University,
  Grenada.
\item
  Koper, N.*, D. Bruinsma, C.M. Curry, L. McDonald, M. Warrington, J.
  Bernath-Plaisted, H. Nenninger, and E. Prokopanko. 2016. If we build
  it, will they come? Manitoba's grassland birds and their native
  prairie habitats. Living Prairie Museum Public Lecture Series,
  Winnipeg, Manitoba, Canada.
\item
  Koper, N.*, C.M. Curry, B. Antze, J. Daniel, H. Nenninger, J.
  Bernath-Plaisted, P. Rosa, and M. Warrington, M. 2016. Effects of
  energy infrastructure on grassland songbirds. Cenovus Lunch and Learn
  seminar series (invited), Cenovus Energy, Calgary, Alberta, Canada.
\item
  Curry, C.M.* 2015. Management and analysis of quantitative data. Guest
  lecture in Master's Thesis Research Seminar (NRI 7262). (Invited by
  the instructor, Dr.~John Sinclair).
\item
  Koper, N.*, J. Bernath-Plaisted, C.M. Curry, B. Antze, M. Warrington,
  H. Nenninger, C. Swider, and P. Rosa. 2015. Effects of energy
  infrastructure and operating noise on grassland songbirds in Alberta.
  National Wildlife Research Centre Seminar Series, National Wildlife
  Research Centre, Carleton University, Ottawa, Canada.
\item
  Koper, N.*, P. Rosa, J. Lockhart, T. Lwiwski, J. Rodgers, K. Molloy,
  C.M. Curry, S. Fischer, C. Swider, and J. Yoo. 2015. From mensurative
  to manipulative: diverse study designs to understand effects of
  anthropogenic disturbance and habitat fragmentation on grassland
  birds. Ontario Grassland Guild: Guelph, ON.
\item
  Koper, N.*, J. Rodgers, J. Yoo, K. Molloy, J. Bernath-Plaisted, C.M.
  Curry, B. Antze, M. Warrington, H. Nenninger, C. Swider, and P. Rosa.
  2015. Effects of shallow gas and oil infrastructure and operating
  noise on grassland songbirds. Canadian Wildlife Service: Edmonton, AB.
\item
  Curry, C.M.* 2014. Adaptation and evolution: the role of evolution in
  conservation and management. Guest lecture in Ecological Dimensions of
  Resource and Environmental Management (NRI7232). Natural Resources
  Institute, University of Manitoba, Winnipeg, MB. (Invited by the
  instructor, Dr.~Nicola Koper).
\item
  Curry, C.M.* 2014. Evolution of reproductive isolation in a temporally
  complex songbird hybrid zone. Seminar at Natural Resources Institute,
  University of Manitoba, Winnipeg, MB. (Oral presentation).
\item
  Curry, C.M.* 2013. The evolution of reproductive isolation in a
  temporally complex passerine hybrid zone. Ecology and Evolutionary
  Biology seminar at Kansas State University: Manhattan, KS. (Invited
  seminar, host Dr.~Brett Sandercock).
\item
  Curry, C.M.* 2013. The evolution of reproductive isolation in a
  temporally complex passerine hybrid zone. Biology department seminar,
  University of Oklahoma: Norman, OK. (Oral presentation).
\item
  Curry, C.M.* and M.A.~Patten. 2011. Song varies across younger and
  older hybrid zones in Black-crested (Baeolophus atricristatus) and
  Tufted (B. bicolor) titmice. Ecology and Evolutionary Biology seminar
  (``Ecomunch'') at University of Oklahoma: Norman, OK. (Oral
  presentation).
\item
  Curry, C.M.* and M.A.~Patten. 2011. Song varies across younger and
  older hybrid zones in Black-crested (Baeolophus atricristatus) and
  Tufted (B. bicolor) titmice. Graduate College Student Research and
  Performance Day, University of Oklahoma: Norman, OK. (Poster
  presentation).
\item
  Curry, C.M.* and M.A.~Patten. 2010. Song differences between Tufted
  and Black-crested Titmouse. Graduate College Student Research and
  Performance Day, University of Oklahoma: Norman, OK. (Poster
  presentation).
\item
  Curry, C.M.* 2009. Genetic and behavioral dynamics in a complex avian
  hybrid zone. Ecology and Evolutionary Biology seminar (``Ecomunch'')
  at University of Oklahoma: Norman, OK. (Oral presentation).
\end{itemize}

\section{Impact}\label{impact}

\subsection{Librarianship}\label{librarianship}

\subsection{Science}\label{science}

\section{Science Outreach}\label{science-outreach}

\subsection{Presentations}\label{presentations}

\begin{itemize}
\tightlist
\item
  Curry, C.M.* 14 Feb.~2019. Can they hear each other now? Noise and its
  effects on birds in the Canadian prairies and elsewhere. Fort Worth
  Audubon Society: Fort Worth, TX. (Public talk).
\item
  Curry, C.M.* 22 Jan.~2018. Can they hear each other now? Noise and its
  effects on birds in the Canadian prairies and elsewhere. Oklahoma City
  Audubon Society: Oklahoma City, OK. (Public talk).
\item
  Curry, C.M.* 2015. Evolution in action in Oklahoma and Texas.
  Presentations via teleconference to high school students in Jenks,
  Oklahoma AP Environmental Science classes (four periods, Mr.~Bryan
  Yockers) about the evolution of reproductive isolation in titmice.
\item
  Curry, C.M.* 2012. Titmice in Oklahoma and Texas: what makes a hybrid
  zone? Oklahoma City Audubon Society: Oklahoma City, OK. (Public talk).
\item
  Curry, C.M.* 2009. Hybrid zones: titmice and the nature of species.
  Darwin-a-thon public education event at University of Oklahoma:
  Norman, OK. (Public talk).
\item
  Have co-presented talks about birds and ecology to Denton County
  Master Gardeners (TX), Elm Fork Chapter of Texas Master Naturalists,
  Decatur Lions' Club, Tallgrass Prairie Audubon Society, Girl Scouts,
  Boy Scouts, elementary school students in Decatur, TX, and 2001, 2006,
  and 2007 JAKES (Juniors Acquiring Knowledge, Ethics, and
  Sportsmanship) events (U.S. Forest Service; Texas Parks and Wildlife)
  at the Lyndon B. Johnson National Grasslands.
\end{itemize}

\subsection{Popular articles}\label{popular-articles}

\begin{itemize}
\tightlist
\item
  Monthly ``Birds and Beyond'' column for a small Texas newspaper, the
  Wise County Messenger. September 2002-September 2007, volunteer
  columnist; November 2007-October 2013, paid columnist. Articles
  generally focused on natural history and identification of
  north-central Texas birds, butterflies, plants, and other organisms
  and descriptions of trips in other areas of Texas and Oklahoma.
\item
  Curry, Claire. 2004. West Nile virus: birds, people, and Texas.''
  Texas Birds 5(1):14-16.

  \begin{itemize}
  \tightlist
  \item
    A short review of West Nile Virus after its initial outbreak.
  \end{itemize}
\item
  Curry, Claire. 2004. Killer meadowlarks. Texas Birds 5(1):29-30.

  \begin{itemize}
  \tightlist
  \item
    Observations on an Eastern Meadowlark killing American Goldfinches
    during a winter storm.
  \end{itemize}
\item
  Curry, Claire. 2004. Flight path: Young Birder of the Year contest
  \textasciitilde{} A birding essay: Stubs. Birding August
  36(4):354-356.

  \begin{itemize}
  \tightlist
  \item
    Observations on a female Archilochus sp. hummingbird with a broken
    beak.
  \end{itemize}
\item
  Curry, Claire. 2004. Real birders drool. IN Claire Curry, Robert M.
  Milardo, and Richard Frechette. 2004. Dimensions: Three flights of
  fancy. Birding 36(3):288-289.

  \begin{itemize}
  \tightlist
  \item
    Humorous essay on the canine birding experience.
  \end{itemize}
\item
  Curry, Claire. 2002. The tail of an unfortunate chickadee. Winging It
  14(2):4.

  \begin{itemize}
  \tightlist
  \item
    Description of a Carolina Chickadee trapped in an Argiope sp. spider
    web.
  \end{itemize}
\end{itemize}

\bookmarksetup{startatroot}

\chapter{Service}\label{service}

\section{Departmental}\label{departmental}

\begin{itemize}
\tightlist
\item
  Hiring committee (chair) for University Libraries STEM Instruction
  Librarian -- 2024 (in progress)
\item
  Hiring committee for University Libraries National Weather Center
  Branch Library Manager -- 2023 (successful search)
\item
  UL Awards committee (2023) Hiring committee for University Libraries
  Liaison Coordinator -- 2022-2023 (unsuccessful searches)
\item
  Hiring committee for University Libraries Marketing \& Communication
  Search -- 2021 (successful search)
\item
  Successful proposal (with assistance from LGBTQIA+ colleagues) to add
  option for pronouns to official University Libraries name badges for
  faculty/staff - 2021
\item
  Department of Biology Graduate Selections Committee graduate student
  representative (2010-2013 academic years); provide comments on new
  graduate student applications to the department.
\item
  Founded and organize Ecology Journal Club at OU with weekly
  discussions by students and faculty of published papers (fall
  2011-fall 2013; responsibilities passed to a new student during my
  last semester of spring 2014 and meetings continued through 2018).
\end{itemize}

\section{University}\label{university}

\begin{itemize}
\tightlist
\item
  Institutional Animal Care and Use Committee Scientist Member, February
  2021-present
\item
  OU Coding Outreach for Data Education (CODE) Workshop helper for R and
  Python: Fall 2020, Fall 2021, Fall 2022.
\item
  Judge for OU Honors College Undergraduate Research Day, April 2020.
\item
  Judge for OU Honors College Undergraduate Research Day, April 2019.
\item
  Reviewer for OU School of Architecture Program for Research
  Enhancement to provide layperson's feedback: fall 2018, spring 2019,
  fall 2019.
\item
  Software and Data Carpentry helper and instructor: 2017-present.
\item
  Guest panelist in Earth Observation Science for Society and
  Sustainability (EOS 3) interdisciplinary certificate program class on
  ``Building a better Ph.D.~or M.S.'' August 28, 2017. Class taught by
  Drs. Jeff Kelly, Andrea Contina, and Sara Mata, University of
  Oklahoma, Norman, OK, USA.
\item
  Nominated to serve as a OU Graduate College Academic Appeals and
  Misconduct student panelist (2011-2012 academic year); nominated by
  the OU Department of Zoology's Graduate Liaison for being ``an
  experienced Graduate Student and hav{[}ing{]} demonstrated the
  qualities necessary to serve as an Appeals or Misconduct panelist such
  as judgment, discretion and commitment.''
\end{itemize}

\section{Peer reviewer}\label{peer-reviewer}

\href{https://publons.com/a/1187905/}{Publons profile (external link)}

\begin{itemize}
\tightlist
\item
  Auk: Ornithological Advances
\item
  Avian Conservation and Ecology
\item
  Condor (now Ornithological Applications)

  \begin{itemize}
  \tightlist
  \item
    Member of the Cooper Ornithological Society's Early Career Editorial
    Board for Condor (2011-2012; graduated from the board at the end of
    that period).
  \end{itemize}
\item
  Diversity and Distributions
\item
  Ecology and Evolution
\item
  Environmental Pollution
\item
  Evidence-based Toxicology
\item
  Ibis
\item
  Journal of Applied Ecology
\item
  Northeastern Naturalist
\item
  Rangeland Ecology and Management
\item
  Wilson Journal of Ornithology
\end{itemize}

\section{External}\label{external}

\begin{itemize}
\tightlist
\item
  Instructor on TAGS (Geolocator Data Cleaning website) at Geolocation
  Workshop August 18-19, 2018. Preceding International Ornithological
  Congress: Vancouver, BC, Canada.
\item
  Session Chair (``Evolution''), 2018 American Ornithology Conference:
  Tuscon, AZ, USA.
\item
  Judged student presentations (oral presentations and posters) at 2018
  American Ornithology Conference: Tuscon, AZ, USA.
\item
  Session Chair (``Landscape ecology''), 2017 American Ornithology
  Conference: East Lansing, MI, USA.
\item
  Judged student presentations (oral presentations and posters) at 2017
  American Ornithology Conference: East Lansing, MI, USA.
\item
  Judged undergraduate presentations (oral presentations and posters) at
  North American Ornithological Conference 2016: Washington, DC., USA.
\item
  Presented and led discussion on project and risk management during
  Critical Skills for the Early Career Professional workshop at 2015 The
  Wildlife Society Meeting, Winnipeg, MB, October 2015.
\item
  Panelist on careers in biology at Prairie University Biology
  Symposium, Winnipeg, MB, February 2015; one of three early career
  biologists answering questions from graduate students.
\item
  Undergraduate Diversity Mentor for two undergraduates attending the
  Evolution 2013 meeting in Snowbird, UT.
\item
  Session Chair (``Diseases/parasites''), NAOC-V meeting, 14-18 August
  2012, Vancouver BC.
\item
  Volunteer, Evolution 2011 meeting, 17-21 July, Norman OK; driving van.
\item
  Session Chair (``Diversification- Birds''), Evolution 2011 meeting,
  17-21 July 2011, Norman OK.
\end{itemize}

\section{Community}\label{community}

\begin{itemize}
\tightlist
\item
  Co-led nature walk for National Trails Day at Fort Richardson State
  Park, TX, with park manager in June 2012.
\item
  Expert biologist at 2011 BioBlitz Oklahoma at Chickasaw National
  Recreation Area, Murray County, OK Guided bird walk for Fort Griffin
  State Historic Site, TX, June 2011.
\item
  Co-led monthly field trips to the Lyndon B. Johnson National
  Grasslands from February 2004 to January 2005.
\item
  Compiled checklists of Wise County birds and butterflies for
  distribution at the U.S. Forest Service office in Decatur, TX.
\item
  Participated in Wise County Christmas Bird Count circle from
  2000-2007.
\item
  Participated in Project Prairie Birds in winters of 2002-2006 on the
  Lyndon B. Johnson National Grasslands.
\item
  Co-editor of Tall Grass Tales, newsletter of Tallgrass Prairie Audubon
  Society, from December 2001 to October 2004.
\item
  Assisted in preparing bird specimens and mounts for display at the
  U.S. Forest Service office in Decatur, TX.
\item
  Completed volunteer hours and advanced training hours for Elm Fork
  chapter of Texas Master Naturalist program, Class of 2000. Certified
  from 2000 to 2006; ended service to focus on undergraduate degree.
\end{itemize}

\bookmarksetup{startatroot}

\chapter{Awards, Honors, and
Fellowships}\label{awards-honors-and-fellowships}

\section{Thomas \& Catherine Luccock Library Award of
Excellence}\label{thomas-catherine-luccock-library-award-of-excellence}

\marginnote{\begin{footnotesize}

May 2022

Section~\ref{sec-science-librarian}

\end{footnotesize}}

University of Oklahoma Libraries ``The Thomas \& Catherine Luccock
Library Award of Excellence is the highest honor awarded by the
University of Oklahoma Libraries. This prestigious award will be
presented annually to recognize and commend an individual for superior
performance and contributions to the University of Oklahoma Libraries.''

\section{Hidden Hero Award University of Oklahoma
Libraries}\label{hidden-hero-award-university-of-oklahoma-libraries}

\marginnote{\begin{footnotesize}

September 2021

Section~\ref{sec-science-librarian}

\end{footnotesize}}

``Going above and beyond in the performance of your job duties.''

\section{Collaborative Spirit Award (for STEM Services team) University
of Oklahoma
Libraries}\label{collaborative-spirit-award-for-stem-services-team-university-of-oklahoma-libraries}

\marginnote{\begin{footnotesize}

April 2020

Section~\ref{sec-science-librarian}

\end{footnotesize}}

``Recognizes a work team or volunteer group that successfully works
together to create a product or result that was strengthened by the
effort of the team.''

\section{Elective Member American Ornithology
Society}\label{sec-aos-elective}

\marginnote{\begin{footnotesize}

August 2017

Section~\ref{sec-peer}

\end{footnotesize}}

Elected to honorary membership category recognizing contributions to
ornithology.

\section{GAANN (Graduate Assistance in Areas of National Need) Fellow
Department of
Education}\label{gaann-graduate-assistance-in-areas-of-national-need-fellow-department-of-education}

\marginnote{\begin{footnotesize}

August 2012-March 2014 (graduation)

\end{footnotesize}}

Nominated by department; covers stipend, tuition, fees

\section{Alumni Fellow University of Oklahoma Graduate
College}\label{alumni-fellow-university-of-oklahoma-graduate-college}

\marginnote{\begin{footnotesize}

2008-2014

\end{footnotesize}}

``emphasizes the recruitment and retention of outstanding graduate
students''. More information at
\url{http://www.ou.edu/gradweb/funding_and_aid/fellowships.html}.

\section{Biology Department Award for Excellence in Graduate Student
Teaching}\label{biology-department-award-for-excellence-in-graduate-student-teaching}

\marginnote{\begin{footnotesize}

May 2013

\end{footnotesize}}

University of Oklahoma Department of Biology Nominated for teaching
Ecology (Section~\ref{sec-teach-ecology}) and Entomology
(Section~\ref{sec-teach-ento}) labs.

\section{Best oral presentation by a graduate
student}\label{best-oral-presentation-by-a-graduate-student}

\marginnote{\begin{footnotesize}

October 2012

\end{footnotesize}}

Oklahoma Ornithological Society Fall 2012 technical meeting

\section{President's List University of North
Texas}\label{presidents-list-university-of-north-texas}

\marginnote{\begin{footnotesize}

Spring 2005-Fall 2007

(all semesters enrolled)

\end{footnotesize}}

For 4.00 GPA while completing at least 12 credit hours in a semester

\section{National Grasslands Prairie Partner
Award}\label{national-grasslands-prairie-partner-award}

\marginnote{\begin{footnotesize}

2003

\end{footnotesize}}

(shared with M. Curry) National Grasslands Council For volunteer work at
the Lyndon B. Johnson National Grasslands.

\bookmarksetup{startatroot}

\chapter{Successful Funding}\label{successful-funding}

\section{Intramural}\label{intramural}

\subsection{Research and Projects}\label{research-and-projects}

\subsubsection{OU Data Institute for Societal Challenges Seed
Funding}\label{ou-data-institute-for-societal-challenges-seed-funding}

\$10,000

\marginnote{\begin{footnotesize}

June 2022-August 2023

\end{footnotesize}}

Seed funding to develop Statistics Helper educational website prototype

\subsubsection{M. Blanche Adams and M. Frances Adams Memorial
Scholarships in Zoology: Summer Research
Scholarship}\label{m.-blanche-adams-and-m.-frances-adams-memorial-scholarships-in-zoology-summer-research-scholarship}

\$2500

\marginnote{\begin{footnotesize}

Summer 2012

\end{footnotesize}}

Proposal title ``Evolution of reproductive isolation in a spatially and
temporally complex songbird hybrid zone''; summer research assistantship
in department

\subsubsection{George Miksch Sutton Scholarship in
Ornithology}\label{george-miksch-sutton-scholarship-in-ornithology}

\$3500

\marginnote{\begin{footnotesize}

2012

\end{footnotesize}}

Proposal title ``Testing hybrid zone models using the Tufted (Baeolophus
bicolor) and Black-crested (B. atricristatus) Titmouse hybrid zone'',
research equipment, travel, and lab expenses

\subsubsection{M. Blanche Adams and M. Frances Adams Memorial
Scholarships in Zoology: Academic Year
Scholarship}\label{m.-blanche-adams-and-m.-frances-adams-memorial-scholarships-in-zoology-academic-year-scholarship}

\$1000

\marginnote{\begin{footnotesize}

2011-2012

\end{footnotesize}}

Proposal title ``Behavioral and genetic dynamics of a complex avian
hybrid zone''; for professional travel and research expenses

\subsubsection{M. Blanche Adams and M. Frances Adams Memorial
Scholarships in Zoology: Summer Research
Scholarship}\label{m.-blanche-adams-and-m.-frances-adams-memorial-scholarships-in-zoology-summer-research-scholarship-1}

\$2500

\marginnote{\begin{footnotesize}

Summer 2011

\end{footnotesize}}

Proposal title ``Behavioral and genetic dynamics of a complex avian
hybrid zone''; summer research assistantship in department

\subsubsection{George Miksch Sutton Scholarship in
Ornithology}\label{george-miksch-sutton-scholarship-in-ornithology-1}

\$6700

\marginnote{\begin{footnotesize}

2011

\end{footnotesize}}

Proposal title ``Testing hybrid zone models using the Tufted (Baeolophus
bicolor) and Black-crested (B. atricristatus) Titmouse hybrid zone'',
research equipment, travel, and lab expenses

\subsubsection{M. Blanche Adams and M. Frances Adams Memorial
Scholarships in Zoology: Academic Year
Scholarship}\label{m.-blanche-adams-and-m.-frances-adams-memorial-scholarships-in-zoology-academic-year-scholarship-1}

\$1000

\marginnote{\begin{footnotesize}

2010-2011

\end{footnotesize}}

Proposal title ``Behavioral and genetic dynamics of a complex avian
hybrid zone''; for professional travel and research expenses

\subsubsection{M. Blanche Adams and M. Frances Adams Memorial
Scholarships in Zoology: Summer Research
Scholarship}\label{m.-blanche-adams-and-m.-frances-adams-memorial-scholarships-in-zoology-summer-research-scholarship-2}

\$756

\marginnote{\begin{footnotesize}

summer 2010

\end{footnotesize}}

Proposal title ``Behavioral and genetic dynamics of a complex avian
hybrid zone''; partial summer research assistantship in department
(complements 2010 Sutton Scholarship stipend, below)

\subsubsection{George Miksch Sutton Scholarship in
Ornithology}\label{george-miksch-sutton-scholarship-in-ornithology-2}

\$4800

\marginnote{\begin{footnotesize}

2010

\end{footnotesize}}

Proposal title ``Dynamics of a complex avian hybrid zone'', research
equipment, travel, lab expenses, and partial summer stipend

\subsubsection{OU Graduate Student Senate Research and Creative
Endeavors
Grant}\label{ou-graduate-student-senate-research-and-creative-endeavors-grant}

\$224.82

\marginnote{\begin{footnotesize}

Spring 2010

\end{footnotesize}}

Research equipment and travel to north-central Texas to examine field
sites

\subsubsection{OU Robberson Research
Grant}\label{ou-robberson-research-grant}

\$1000 (granted once per degree)

\marginnote{\begin{footnotesize}

February 2010

\end{footnotesize}}

Research travel and genetic analysis costs

\subsubsection{OU Graduate Student Senate Research and Creative
Endeavors
Grant}\label{ou-graduate-student-senate-research-and-creative-endeavors-grant-1}

\$315.62

\marginnote{\begin{footnotesize}

Fall 2009

\end{footnotesize}}

Travel expenses to southwestern Oklahoma to examine field sites

\subsubsection{M. Blanche Adams and M. Frances Adams Memorial
Scholarships in Zoology: Summer Research
Scholarship}\label{m.-blanche-adams-and-m.-frances-adams-memorial-scholarships-in-zoology-summer-research-scholarship-3}

\$2500

\marginnote{\begin{footnotesize}

Summer 2009

\end{footnotesize}}

Proposal title ``Behavioral and genetic dynamics of a complex avian
hybrid zone''; summer research assistantship in department

\subsubsection{George Miksch Sutton Scholarship in
Ornithology}\label{george-miksch-sutton-scholarship-in-ornithology-3}

\$5000

\marginnote{\begin{footnotesize}

2009

\end{footnotesize}}

Proposal title ``Dynamics of a complex avian hybrid zone''; research
equipment and travel expenses

\subsection{Travel Awards}\label{travel-awards}

\subsubsection{OU College of Arts and Sciences Student Travel
Fund}\label{ou-college-of-arts-and-sciences-student-travel-fund}

\$750

\marginnote{\begin{footnotesize}

August 2013

\end{footnotesize}}

Travel to joint meeting in Chicago, IL, Utah in August 2013, \$750.

\subsubsection{OU Department of Biology Travel
Funding}\label{ou-department-of-biology-travel-funding}

\$250

\marginnote{\begin{footnotesize}

May 2013

\end{footnotesize}}

To attend Evolution 2013 meeting in Snowbird, UT in June 2013

\subsubsection{OU College of Arts and Sciences Student Travel
Fund}\label{ou-college-of-arts-and-sciences-student-travel-fund-1}

\$750

\marginnote{\begin{footnotesize}

August 2012

\end{footnotesize}}

Travel to North American Ornithological Conference in Vancouver BC in
August 2012

\subsubsection{OU Department of Zoology travel
funding}\label{ou-department-of-zoology-travel-funding}

\$250

\marginnote{\begin{footnotesize}

February 2011

\end{footnotesize}}

Travel to Association of Field Ornithologists/Cooper Ornithological
Society/Wilson Ornithological Society joint meeting in Kearney, Nebraska
in March 2011

\subsubsection{OU College of Arts and Sciences Student Travel
Fund}\label{ou-college-of-arts-and-sciences-student-travel-fund-2}

\$750

\marginnote{\begin{footnotesize}

October 2010

\end{footnotesize}}

Travel to Association of Field Ornithologists' conference in Ogden, Utah
in August 2010

\subsubsection{OU Robberson Travel
Grant}\label{ou-robberson-travel-grant}

\$500 (granted once per degree)

\marginnote{\begin{footnotesize}

October 2010

\end{footnotesize}}

Travel to Association of Field Ornithologists' conference in Ogden, Utah
in August 2010

\subsubsection{OU Graduate Student Senate Conference and Creative
Exhibition
Grant}\label{ou-graduate-student-senate-conference-and-creative-exhibition-grant}

\$79.86

\marginnote{\begin{footnotesize}

Spring 2010

\end{footnotesize}}

Lodging and daily expenses at Southwestern Association of Naturalists'
meeting in Junction, TX

\subsubsection{Carl and Pat Bynum Riggs
Scholarship}\label{carl-and-pat-bynum-riggs-scholarship}

\$500

\marginnote{\begin{footnotesize}

Summer 2009

\end{footnotesize}}

To defray expenses of course tuition and fees for O.U. Biological
Station summer session; Molecular Techniques for Field Biology class

\section{Extramural}\label{extramural}

\subsection{Research}\label{research}

\subsubsection{Prairie Biotic Research, Inc.~small grants
program}\label{prairie-biotic-research-inc.-small-grants-program}

\$1500

\marginnote{\begin{footnotesize}

March 2018

\end{footnotesize}}

Proposal title, ``An Illustrated Checklist of the Lichens of Wise
County, Texas, and Surrounding Areas''

Co-PI Mary Curry

\subsubsection{Amazon Web Services (AWS) Cloud Credits for Research
program}\label{amazon-web-services-aws-cloud-credits-for-research-program}

\$6500 in computing credits on AWS

\marginnote{\begin{footnotesize}

2017

\end{footnotesize}}

Proposal title ``Predictive accuracy and efficiency of spatially
explicit species distribution models''

\subsubsection{Sigma Xi Grant-in-Aid of
Research}\label{sigma-xi-grant-in-aid-of-research}

\$500

\marginnote{\begin{footnotesize}

Spring 2012

\end{footnotesize}}

Proposal title ``Evolution of reproductive isolation in a spatially and
temporally complex songbird hybrid zone'', travel mileage to field sites

\subsubsection{American Museum of Natural History Frank M. Chapman
Memorial Fund
Grant}\label{american-museum-of-natural-history-frank-m.-chapman-memorial-fund-grant}

\$2000

\marginnote{\begin{footnotesize}

2012-2013

\end{footnotesize}}

Proposal title ``Shadow of a doubt: species recognition in a complex
avian hybrid zone'', research expenses and genetic analysis costs

\subsubsection{American Museum of Natural History Frank M. Chapman
Memorial Fund
Grant}\label{american-museum-of-natural-history-frank-m.-chapman-memorial-fund-grant-1}

\$2000

\marginnote{\begin{footnotesize}

2011-2013

\end{footnotesize}}

Proposal title ``Why songs change over space and time in a complex avian
hybrid zone'', research expenses and genetic analysis costs

\subsubsection{Oklahoma Ornithological
Society}\label{oklahoma-ornithological-society}

\$500

\marginnote{\begin{footnotesize}

2012

\end{footnotesize}}

Proposal title ``Shadow of a doubt: species recognition and fitness in a
complex avian hybrid zone'', research expenses

\subsubsection{Oklahoma Ornithological
Society}\label{oklahoma-ornithological-society-1}

\$400

\marginnote{\begin{footnotesize}

2010

\end{footnotesize}}

Proposal title ``Genetic and vocal dynamics of a complex avian hybrid
zone'', research expenses

\subsection{Travel Awards}\label{travel-awards-1}

\subsubsection{American Ornithological Society Post-Doctoral Travel
Award}\label{american-ornithological-society-post-doctoral-travel-award}

\$290

\marginnote{\begin{footnotesize}

August 2017

\end{footnotesize}}

To attend 2017 American Ornithology Conference in East Lansing, MI

\subsubsection{Association of Field Ornithologists' Post-Doctoral Travel
Award}\label{association-of-field-ornithologists-post-doctoral-travel-award}

\$260

\marginnote{\begin{footnotesize}

August 2016

\end{footnotesize}}

To attend NAOC2016 meeting in Washington, DC

\subsubsection{American Ornithologists Union/Cooper Ornithological
Society/Society of Canadian Ornithologists Joint Student Travel
Award}\label{american-ornithologists-unioncooper-ornithological-societysociety-of-canadian-ornithologists-joint-student-travel-award}

\$370

\marginnote{\begin{footnotesize}

September 2014

\end{footnotesize}}

To attend AOU/COS/SCO joint meeting in Estes Park, CO

\subsubsection{American Ornithologists Union/Cooper Ornithological
Society Student Travel
Award}\label{american-ornithologists-unioncooper-ornithological-society-student-travel-award}

\$333

\marginnote{\begin{footnotesize}

August 2013

\end{footnotesize}}

To attend AOU/COS joint meeting in Chicago, IL

\subsubsection{American Society of Naturalists Student Travel
Award}\label{american-society-of-naturalists-student-travel-award}

\$500

\marginnote{\begin{footnotesize}

June 2013

\end{footnotesize}}

To attend Evolution 2013 meeting in Snowbird, UT

\subsubsection{Student Travel Award from 2012 North American
Ornithological
Conference}\label{student-travel-award-from-2012-north-american-ornithological-conference}

\$300

\marginnote{\begin{footnotesize}

2012

\end{footnotesize}}

To attend 2012 North American Ornithological Conference in Vancouver, BC

\bookmarksetup{startatroot}

\chapter{Professional Memberships}\label{professional-memberships}

\section{American Statistical
Association}\label{american-statistical-association}

\marginnote{\begin{footnotesize}

2022-present

\end{footnotesize}}

Education Section and Oklahoma Chapter

\section{American Ornithological
Society}\label{american-ornithological-society}

\marginnote{\begin{footnotesize}

2010-present

\end{footnotesize}}

Voted Elective Member in 2017 (Section~\ref{sec-aos-elective})

\section{Association of Field
Ornithologists}\label{association-of-field-ornithologists}

\marginnote{\begin{footnotesize}

2012-2022

\end{footnotesize}}

Member

\section{Research Data Preservation
Association}\label{research-data-preservation-association}

\marginnote{\begin{footnotesize}

2018-2019

\end{footnotesize}}

Member

\bookmarksetup{startatroot}

\chapter{Training and Skills}\label{training-and-skills}

\section{Skills}\label{skills}

\subsection{Evidence synthesis}\label{evidence-synthesis}

\begin{itemize}
\tightlist
\item
  Evidence Synthesis Institute. Using training to assist OU community
  members from departments of Social Work, Mathematics, Health and
  Exercise Science, Geography \& Environmental Sustainability, Education
\end{itemize}

\marginnote{\begin{footnotesize}

August 2022 cohort

\end{footnotesize}}

\begin{itemize}
\tightlist
\item
  \href{https://guides.ou.edu/data_resources/synthesis}{Development of
  Evidence Synthesis LibGuide Page (external link)} for initial
  information for community members: Collaborating with Writing Center
  on a future questionnaire to guide users to specific types of
  synthesis.
\item
  Two peer-reviewed narrative reviews (Section~\ref{sec-peer})
\end{itemize}

\subsection{Certified Carpentries
instructor}\label{certified-carpentries-instructor}

Have taught modules on Git/GitHub, Bash, OpenRefine, Tidy Data, R (both
Software and Data Carpentry versions), and helped with Python

\subsection{Working with People and
Organizations}\label{working-with-people-and-organizations}

\begin{itemize}
\tightlist
\item
  Consultations for undergraduates, graduate students, post-doctoral
  researchers, staff, faculty, and community members
\item
  Collaborating with partner organizations and co-workers
\item
  Maintaining good relationships for permission for field work on
  private (US and Canada), state (US), and federal (US) land
\item
  Obtaining permission for field work on private (US), state (US), and
  federal (US) land
\item
  Permitting for avian research at university, state, provincial, and
  federal (US and Canada) levels
\end{itemize}

\subsection{Statistical Advising}\label{statistical-advising}

\begin{itemize}
\tightlist
\item
  Univariate statistics

  \begin{itemize}
  \tightlist
  \item
    Generalized linear models (GLM, ANOVA, ANCOVA)
  \item
    Mixed model/hierarchical generalized linear models
  \end{itemize}
\item
  Multivariate statistics

  \begin{itemize}
  \tightlist
  \item
    Decision trees
  \item
    Random forest
  \item
    Principal components analysis (PCA) and choice of ordination
    techniques
  \item
    Canonical correlation analysis
  \item
    Profile analysis
  \end{itemize}
\end{itemize}

\section{Software}\label{software}

\subsection{Statistical Programs}\label{statistical-programs}

\begin{itemize}
\tightlist
\item
  R 4.0+
\item
  SAS 9.2+
\item
  SPSS
\end{itemize}

\subsection{Geographic Information Systems
(GIS)}\label{geographic-information-systems-gis}

\begin{itemize}
\tightlist
\item
  R for GIS: packages `rgdal', `sp', `sf', `terra', `raster', and
  `rgeos'
\item
  QuantumGIS opensource GIS software v. 2.10+
\item
  ESRI ArcGIS 9.2 and later

  \begin{itemize}
  \tightlist
  \item
    Good relationship with vendor representatives at OU.
  \item
    Current use for consultations
  \item
    Past use for research
  \item
    Assisting users with migration to ArcGIS Pro from ArcMap
  \item
    Completed course Introduction to Geographic Information Systems
    course (UNT, Fall 2007)
  \end{itemize}
\end{itemize}

\subsection{Graphics and Data
Visualization}\label{graphics-and-data-visualization}

\begin{itemize}
\tightlist
\item
  R package \texttt{ggplot2} and extensions such as \texttt{ggparty}
\item
  RShiny
\item
  Inkscape 0.48+ (creating vector graphics)
\end{itemize}

\subsection{Data management and
processing}\label{data-management-and-processing}

\subsubsection{Data wrangling}\label{data-wrangling}

\begin{itemize}
\tightlist
\item
  R package `tidyverse'
\item
  R package `sqldf'
\item
  OpenRefine
\end{itemize}

\subsubsection{Version control}\label{version-control}

\begin{itemize}
\tightlist
\item
  Git
\item
  GitHub.com
\item
  Git Large File System
\end{itemize}

\subsubsection{Databases}\label{databases}

\begin{itemize}
\tightlist
\item
  TODO the db in carpentries
\item
  PostgreSQL database language (queries, import and export of data)
\item
  pgAdmin III database interface and psql command line interface for use
  with PostgreSQL
\item
  LibreOffice 3.4+ Base
\item
  Microsoft Access (uses jetSQL)
\end{itemize}

\subsection{Citation/reference
management}\label{citationreference-management}

\begin{itemize}
\tightlist
\item
  Zotero 3.0.7+ bibliography software
\end{itemize}

\subsection{Word Processing and
Spreadsheets}\label{word-processing-and-spreadsheets}

\begin{itemize}
\tightlist
\item
  Microsoft Office 365: Word, Access, Excel, PowerPoint, OneNote
\item
  LibreOffice 3.3.0+: Writer, Impress (PowerPoint equivalent), Base
  (Access equivalent)
\end{itemize}

\subsection{Specialized Software}\label{specialized-software}

\begin{itemize}
\tightlist
\item
  Raven Pro 1.5 sound analysis software
\item
  Phylogenetic software (basic procedures): MRBAYES 3.1.2, RAxML, GARLI
  1.0, FigTree, and MEGA 4.0
\end{itemize}



\end{document}
